% Options for packages loaded elsewhere
% Options for packages loaded elsewhere
\PassOptionsToPackage{unicode}{hyperref}
\PassOptionsToPackage{hyphens}{url}
\PassOptionsToPackage{dvipsnames,svgnames,x11names}{xcolor}
%
\documentclass[
  letterpaper,
  DIV=11,
  numbers=noendperiod]{scrartcl}
\usepackage{xcolor}
\usepackage{amsmath,amssymb}
\setcounter{secnumdepth}{5}
\usepackage{iftex}
\ifPDFTeX
  \usepackage[T1]{fontenc}
  \usepackage[utf8]{inputenc}
  \usepackage{textcomp} % provide euro and other symbols
\else % if luatex or xetex
  \usepackage{unicode-math} % this also loads fontspec
  \defaultfontfeatures{Scale=MatchLowercase}
  \defaultfontfeatures[\rmfamily]{Ligatures=TeX,Scale=1}
\fi
\usepackage{lmodern}
\ifPDFTeX\else
  % xetex/luatex font selection
\fi
% Use upquote if available, for straight quotes in verbatim environments
\IfFileExists{upquote.sty}{\usepackage{upquote}}{}
\IfFileExists{microtype.sty}{% use microtype if available
  \usepackage[]{microtype}
  \UseMicrotypeSet[protrusion]{basicmath} % disable protrusion for tt fonts
}{}
\makeatletter
\@ifundefined{KOMAClassName}{% if non-KOMA class
  \IfFileExists{parskip.sty}{%
    \usepackage{parskip}
  }{% else
    \setlength{\parindent}{0pt}
    \setlength{\parskip}{6pt plus 2pt minus 1pt}}
}{% if KOMA class
  \KOMAoptions{parskip=half}}
\makeatother
% Make \paragraph and \subparagraph free-standing
\makeatletter
\ifx\paragraph\undefined\else
  \let\oldparagraph\paragraph
  \renewcommand{\paragraph}{
    \@ifstar
      \xxxParagraphStar
      \xxxParagraphNoStar
  }
  \newcommand{\xxxParagraphStar}[1]{\oldparagraph*{#1}\mbox{}}
  \newcommand{\xxxParagraphNoStar}[1]{\oldparagraph{#1}\mbox{}}
\fi
\ifx\subparagraph\undefined\else
  \let\oldsubparagraph\subparagraph
  \renewcommand{\subparagraph}{
    \@ifstar
      \xxxSubParagraphStar
      \xxxSubParagraphNoStar
  }
  \newcommand{\xxxSubParagraphStar}[1]{\oldsubparagraph*{#1}\mbox{}}
  \newcommand{\xxxSubParagraphNoStar}[1]{\oldsubparagraph{#1}\mbox{}}
\fi
\makeatother

\usepackage{color}
\usepackage{fancyvrb}
\newcommand{\VerbBar}{|}
\newcommand{\VERB}{\Verb[commandchars=\\\{\}]}
\DefineVerbatimEnvironment{Highlighting}{Verbatim}{commandchars=\\\{\}}
% Add ',fontsize=\small' for more characters per line
\usepackage{framed}
\definecolor{shadecolor}{RGB}{241,243,245}
\newenvironment{Shaded}{\begin{snugshade}}{\end{snugshade}}
\newcommand{\AlertTok}[1]{\textcolor[rgb]{0.68,0.00,0.00}{#1}}
\newcommand{\AnnotationTok}[1]{\textcolor[rgb]{0.37,0.37,0.37}{#1}}
\newcommand{\AttributeTok}[1]{\textcolor[rgb]{0.40,0.45,0.13}{#1}}
\newcommand{\BaseNTok}[1]{\textcolor[rgb]{0.68,0.00,0.00}{#1}}
\newcommand{\BuiltInTok}[1]{\textcolor[rgb]{0.00,0.23,0.31}{#1}}
\newcommand{\CharTok}[1]{\textcolor[rgb]{0.13,0.47,0.30}{#1}}
\newcommand{\CommentTok}[1]{\textcolor[rgb]{0.37,0.37,0.37}{#1}}
\newcommand{\CommentVarTok}[1]{\textcolor[rgb]{0.37,0.37,0.37}{\textit{#1}}}
\newcommand{\ConstantTok}[1]{\textcolor[rgb]{0.56,0.35,0.01}{#1}}
\newcommand{\ControlFlowTok}[1]{\textcolor[rgb]{0.00,0.23,0.31}{\textbf{#1}}}
\newcommand{\DataTypeTok}[1]{\textcolor[rgb]{0.68,0.00,0.00}{#1}}
\newcommand{\DecValTok}[1]{\textcolor[rgb]{0.68,0.00,0.00}{#1}}
\newcommand{\DocumentationTok}[1]{\textcolor[rgb]{0.37,0.37,0.37}{\textit{#1}}}
\newcommand{\ErrorTok}[1]{\textcolor[rgb]{0.68,0.00,0.00}{#1}}
\newcommand{\ExtensionTok}[1]{\textcolor[rgb]{0.00,0.23,0.31}{#1}}
\newcommand{\FloatTok}[1]{\textcolor[rgb]{0.68,0.00,0.00}{#1}}
\newcommand{\FunctionTok}[1]{\textcolor[rgb]{0.28,0.35,0.67}{#1}}
\newcommand{\ImportTok}[1]{\textcolor[rgb]{0.00,0.46,0.62}{#1}}
\newcommand{\InformationTok}[1]{\textcolor[rgb]{0.37,0.37,0.37}{#1}}
\newcommand{\KeywordTok}[1]{\textcolor[rgb]{0.00,0.23,0.31}{\textbf{#1}}}
\newcommand{\NormalTok}[1]{\textcolor[rgb]{0.00,0.23,0.31}{#1}}
\newcommand{\OperatorTok}[1]{\textcolor[rgb]{0.37,0.37,0.37}{#1}}
\newcommand{\OtherTok}[1]{\textcolor[rgb]{0.00,0.23,0.31}{#1}}
\newcommand{\PreprocessorTok}[1]{\textcolor[rgb]{0.68,0.00,0.00}{#1}}
\newcommand{\RegionMarkerTok}[1]{\textcolor[rgb]{0.00,0.23,0.31}{#1}}
\newcommand{\SpecialCharTok}[1]{\textcolor[rgb]{0.37,0.37,0.37}{#1}}
\newcommand{\SpecialStringTok}[1]{\textcolor[rgb]{0.13,0.47,0.30}{#1}}
\newcommand{\StringTok}[1]{\textcolor[rgb]{0.13,0.47,0.30}{#1}}
\newcommand{\VariableTok}[1]{\textcolor[rgb]{0.07,0.07,0.07}{#1}}
\newcommand{\VerbatimStringTok}[1]{\textcolor[rgb]{0.13,0.47,0.30}{#1}}
\newcommand{\WarningTok}[1]{\textcolor[rgb]{0.37,0.37,0.37}{\textit{#1}}}

\usepackage{longtable,booktabs,array}
\usepackage{calc} % for calculating minipage widths
% Correct order of tables after \paragraph or \subparagraph
\usepackage{etoolbox}
\makeatletter
\patchcmd\longtable{\par}{\if@noskipsec\mbox{}\fi\par}{}{}
\makeatother
% Allow footnotes in longtable head/foot
\IfFileExists{footnotehyper.sty}{\usepackage{footnotehyper}}{\usepackage{footnote}}
\makesavenoteenv{longtable}
\usepackage{graphicx}
\makeatletter
\newsavebox\pandoc@box
\newcommand*\pandocbounded[1]{% scales image to fit in text height/width
  \sbox\pandoc@box{#1}%
  \Gscale@div\@tempa{\textheight}{\dimexpr\ht\pandoc@box+\dp\pandoc@box\relax}%
  \Gscale@div\@tempb{\linewidth}{\wd\pandoc@box}%
  \ifdim\@tempb\p@<\@tempa\p@\let\@tempa\@tempb\fi% select the smaller of both
  \ifdim\@tempa\p@<\p@\scalebox{\@tempa}{\usebox\pandoc@box}%
  \else\usebox{\pandoc@box}%
  \fi%
}
% Set default figure placement to htbp
\def\fps@figure{htbp}
\makeatother


% definitions for citeproc citations
\NewDocumentCommand\citeproctext{}{}
\NewDocumentCommand\citeproc{mm}{%
  \begingroup\def\citeproctext{#2}\cite{#1}\endgroup}
\makeatletter
 % allow citations to break across lines
 \let\@cite@ofmt\@firstofone
 % avoid brackets around text for \cite:
 \def\@biblabel#1{}
 \def\@cite#1#2{{#1\if@tempswa , #2\fi}}
\makeatother
\newlength{\cslhangindent}
\setlength{\cslhangindent}{1.5em}
\newlength{\csllabelwidth}
\setlength{\csllabelwidth}{3em}
\newenvironment{CSLReferences}[2] % #1 hanging-indent, #2 entry-spacing
 {\begin{list}{}{%
  \setlength{\itemindent}{0pt}
  \setlength{\leftmargin}{0pt}
  \setlength{\parsep}{0pt}
  % turn on hanging indent if param 1 is 1
  \ifodd #1
   \setlength{\leftmargin}{\cslhangindent}
   \setlength{\itemindent}{-1\cslhangindent}
  \fi
  % set entry spacing
  \setlength{\itemsep}{#2\baselineskip}}}
 {\end{list}}
\usepackage{calc}
\newcommand{\CSLBlock}[1]{\hfill\break\parbox[t]{\linewidth}{\strut\ignorespaces#1\strut}}
\newcommand{\CSLLeftMargin}[1]{\parbox[t]{\csllabelwidth}{\strut#1\strut}}
\newcommand{\CSLRightInline}[1]{\parbox[t]{\linewidth - \csllabelwidth}{\strut#1\strut}}
\newcommand{\CSLIndent}[1]{\hspace{\cslhangindent}#1}



\setlength{\emergencystretch}{3em} % prevent overfull lines

\providecommand{\tightlist}{%
  \setlength{\itemsep}{0pt}\setlength{\parskip}{0pt}}



 


\KOMAoption{captions}{tableheading}
\makeatletter
\@ifpackageloaded{caption}{}{\usepackage{caption}}
\AtBeginDocument{%
\ifdefined\contentsname
  \renewcommand*\contentsname{Table of contents}
\else
  \newcommand\contentsname{Table of contents}
\fi
\ifdefined\listfigurename
  \renewcommand*\listfigurename{List of Figures}
\else
  \newcommand\listfigurename{List of Figures}
\fi
\ifdefined\listtablename
  \renewcommand*\listtablename{List of Tables}
\else
  \newcommand\listtablename{List of Tables}
\fi
\ifdefined\figurename
  \renewcommand*\figurename{Figure}
\else
  \newcommand\figurename{Figure}
\fi
\ifdefined\tablename
  \renewcommand*\tablename{Table}
\else
  \newcommand\tablename{Table}
\fi
}
\@ifpackageloaded{float}{}{\usepackage{float}}
\floatstyle{ruled}
\@ifundefined{c@chapter}{\newfloat{codelisting}{h}{lop}}{\newfloat{codelisting}{h}{lop}[chapter]}
\floatname{codelisting}{Listing}
\newcommand*\listoflistings{\listof{codelisting}{List of Listings}}
\makeatother
\makeatletter
\makeatother
\makeatletter
\@ifpackageloaded{caption}{}{\usepackage{caption}}
\@ifpackageloaded{subcaption}{}{\usepackage{subcaption}}
\makeatother
\usepackage{bookmark}
\IfFileExists{xurl.sty}{\usepackage{xurl}}{} % add URL line breaks if available
\urlstyle{same}
\hypersetup{
  pdftitle={Geospatial Analysis of a Changing Road Network},
  pdfauthor={Will jones; Ben Jantzen; Yang Shao},
  colorlinks=true,
  linkcolor={blue},
  filecolor={Maroon},
  citecolor={Blue},
  urlcolor={Blue},
  pdfcreator={LaTeX via pandoc}}


\title{Geospatial Analysis of a Changing Road Network}
\author{Will jones \and Ben Jantzen \and Yang Shao}
\date{2025-11-19}
\begin{document}
\maketitle

\renewcommand*\contentsname{Table of contents}
{
\hypersetup{linkcolor=}
\setcounter{tocdepth}{4}
\tableofcontents
}

\section{Abstract}\label{abstract}

The 2003 study ``How Far to the Nearest Road ?'' published in
\emph{Frontiers in Ecology and The Environment} studied the road network
of the United States (\citeproc{ref-riitters.etal.2003}{Riitters and
Wickham 2003}). Specifically, it split the contiguous US into 30x30 m
grid cells and classified each as being a ``road'' or ``non-road'' cell,
based on if any part of the cell intersects a TIGER/line road feature.
Then, a binned classification was used to measure the distance of every
cell to its nearest road cell.

Our aim is to expand on this methodology using GIS and statistical
methods. Specifically, our objective is to answer the following
questions:

\begin{enumerate}
\def\labelenumi{\arabic{enumi}.}
\tightlist
\item
  By randomly sampling points and calculating straight-line (Euclidean)
  distances, can we measure a more accurate continuous distribution of
  distance to the nearest road?
\item
  Using archived Census Bureau TIGER/Line geographic data, can we
  measure how far the average distance to the nearest road has changed
  in the contiguous US?
\item
  What water bodies have experienced the greatest change and become more
  vulnerable to ecological effects of road construction and operation?
\end{enumerate}

\section{Introduction}\label{introduction}

There are well-documented effects of roads on habitat fragmentation
{[}CITE{]}, plant health {[}CITE{]}, and species diversity {[}CITE{]}.
Native stream invertebrate species may be especially at risk. In fact,
Gál, et al.~found that road crossings had negative effects on the
biodiversity of native invertebrates in Hungary, including species
richness, abundance, and prevalence of protected species
(\citeproc{ref-blanka.etal.2020}{Gál et al. 2020}). Between X and X
year, Y miles of roads have been built, an average of X miles of road
per year {[}CITE{]}. Of course, this period encompasses the time since
the last contiguous United States-level measurement of distances to the
nearest road in 2003 (\citeproc{ref-riitters.etal.2003}{Riitters and
Wickham 2003}). Using GIS methods and a statistical approach based on
the energy distance (\citeproc{ref-szekely.gabor.2004}{Szekely, Rizzo,
and Székely 2004}), we may examine the expansion of the United States
road network for future ecological applications. Additionally, through
measurement of a paired continuous distribution of distances (2000 and
2024), we may observe the magnitude of change in distances, both at the
national scale, and at relevant sub-levels, including watershed, EPA
ecoregion, NLCD landcover class, and even individual stream level.

\section{Methods}\label{methods}

\subsection{Data Acquisition and
Preprocessing}\label{data-acquisition-and-preprocessing}

Six primary spatial datasets were used in our analysis. The 2023 USGS
National Land Cover Database (NLCD), containing 11 distinct land cover
classes, was used to extract surface type at each of our sampled points
(\citeproc{ref-usgs}{{``Annual NLCD Collection 1 Science Products''}
2024}). Likewise, attribute data for each sampled point was extracted
using the EPA's ecoregion level IV map, and watershed membership was
extracted using the USGS 3D Hydrography Program (3DHP) data set. 3DHP is
a unified lidar-derived data product containing vector data for streams,
lakes, ponds, catchments, and other hydrologic features (USGS, 2025).

US Census Bureau's TIGER/Line vector files were used as the road
networks in this project. The most recently available 2024 road network
data for the conterminous United States was downloaded in bulk using the
roads() function from the Python package pygris
(\citeproc{ref-pygris}{{``Pygris,''} n.d.}).

Comparable data for the year 2000 was downloaded from the archived
Census FTP site (\url{https://www2.census.gov/geo/tiger/tiger2k/),} and
a multi- step data preprocessing workflow was performed before the data
was implemented in the analysis workflow. First, beautifulsoup4 was used
to download vector dataset in RT1 format, the historical file format for
census geographic data (\citeproc{ref-beautifulsoup}{Richardson 2007}).
Using GDAL command-line utilities, files were converted to modern
shapefile format, then merged into a single dataset using geopandas in
python (\citeproc{ref-geopandas}{Jordahl et al. 2020}). Finally, feature
correction was performed using geometry-snapping and densification
processing in QGIS 3.36 ''Maidenhead.'' To rectify data quality issues,
namely feature accuracy and spatial mismatch between existing roads in
the 2000 and 2024 datasets, the historical vector dataset was densified
at regular 25m intervals, then snapped to the nearest matching feature
in the current roads dataset. This was done to avoid erroneous
measurement differences in distance to the nearest road for each year
that may be incorrectly attributed to removal or addition of a new road
feature (\citeproc{ref-t2024}{\textbf{t2024?}}).

\subsection{Sampling Scheme and
Measurement}\label{sampling-scheme-and-measurement}

\subsubsection{Terrestrial Points}\label{terrestrial-points}

Using the R package sf, 1,000,000 points were randomly sampled across
our study area (Virginia, USA). Points were sampled uniformly across
space, meaning that everywhere on the continuous surface has equal
likelihood of being sampled (Fotheringham and Rogereson 2008). Next, the
nearest feature from each road network (2000 and 2024) was identified
for each randomly sampled point, and the Euclidean distance between the
two geometries was measured. The result was two continuous
distributions, each containing a measurement at the same point. This is
a case of paired data, analogous to `before and after' measurements.

\subsubsection{Water Points}\label{water-points}

To perform the corresponding process for water features (lakes, ponds,
streams, creeks), a slightly different sampling method was used. It is
our goal to randomly select points along linear water features and along
the \emph{edges} of waterbodies (polygons). We do not want to sample
points within ponds or lakes, because the measurement of interest is the
distance between the least central points of each areal feature (i.e.,
the ``shore'') and the nearest road feature. This effectively captures
where ecological interactions such as runoff into ponds/lakes will
occur. To do this, the layers \texttt{hydro\_3dhp\_all\_flowline} and
\texttt{hydro\_3dhp\_all\_waterbody} were extracted from the USGS 3DHP
geodatabase, representing flowlines and waterbodies, respectively. Due
to the structure of the 3DHP features, \texttt{flowlines} are often
drawn through the waterbody that they feed into. To navigate this, and
avoid sampling within water bodies, the \texttt{sf} package function
\texttt{st\_difference} was used to eliminate these overlapping
flowlines from the features that could be sampled on. Lastly, lake and
pond boundaries were merged with stream and river features into a single
linear feature layer. When using \texttt{st\_sample}, this ensures that
the points are uniformly randomly sampled, meaning that any point in our
dataset has equal probability of being sampled.

\begin{figure}

\begin{minipage}{0.50\linewidth}

\centering{

\pandocbounded{\includegraphics[keepaspectratio]{ProjectFigures/wrong_water_pts.png}}

}

\subcaption{\label{fig-wrong}Sampling on water features before removing
overlapping flowlines}

\end{minipage}%
%
\begin{minipage}{0.50\linewidth}

\centering{

\pandocbounded{\includegraphics[keepaspectratio]{ProjectFigures/good_water_pts.png}}

}

\subcaption{\label{fig-right}Sampling on water features after removing
overlaping flowlines}

\end{minipage}%

\end{figure}%

\subsection{Statistical Analysis}\label{statistical-analysis}

For the purposes of this class project, 100,000 of the
\textasciitilde1e6 points were chosen randomly for analysis in order to
make local processing possible.

To compare the shift in distributions from 2000 to 2024, a permutation
test for matched pairs was used (\citeproc{ref-welch.1990}{Welch 1990}),
along with bootstrap resampling (\citeproc{ref-dixon.2001}{Dixon 2001})
to construct a confidence interval for the test statistic of interest,
the mean of the differences between the paired points. The mean of the
paired differences was chosen as the test statistic for analysis because
the data set is dominated by points whose values did not change (i.e.,
difference is 0).This characteristic of the distribution, along with the
fact that the distribution of differences is not symmetric around the
mean (see Figure~\ref{fig-qq}), is why we did not use the Wilcoxon
Signed Rank test.

\begin{figure}

\centering{

\pandocbounded{\includegraphics[keepaspectratio]{ProjectFigures/differences_qq.png}}

}

\caption{\label{fig-qq}}

\end{figure}%

We used bootstrapping (\citeproc{ref-dixon.2001}{Dixon 2001}) to
construct a sampling distribution and confidence interval for the mean
of paired differences. The observed mean of the differences between our
paired data is 74.89 meters. After bootstrap resampling, we have a
theoretical sampling distribution with a mean of 74.87 meters (95\% CI:
(73.19, 76.60)) and a standard error of 0.87 meters.

\pandocbounded{\includegraphics[keepaspectratio]{ProjectFigures/bootstrap_hist.png}}

\subsubsection{Permutation Test}\label{permutation-test}

Then we performed a permutation test for paired data
(\citeproc{ref-matched.perm}{{``Permutation Test for Matched Pairs
Data,''} n.d.}) to test for significance in our results:

\begin{Shaded}
\begin{Highlighting}[]
\NormalTok{results }\OtherTok{=} \FunctionTok{c}\NormalTok{()}
\ControlFlowTok{for}\NormalTok{ (i }\ControlFlowTok{in} \DecValTok{1}\SpecialCharTok{:}\DecValTok{5000}\NormalTok{) \{}
\NormalTok{  permutedData }\OtherTok{=} \FunctionTok{sample}\NormalTok{(}\FunctionTok{c}\NormalTok{(}\DecValTok{1}\NormalTok{,}\SpecialCharTok{{-}}\DecValTok{1}\NormalTok{),}\DecValTok{100000}\NormalTok{,}\AttributeTok{replace=}\NormalTok{T)}\SpecialCharTok{*}\NormalTok{dat}\SpecialCharTok{$}\NormalTok{difference}
\NormalTok{  results }\OtherTok{=} \FunctionTok{c}\NormalTok{(}\FunctionTok{mean}\NormalTok{(permutedData),results)}
\NormalTok{\}}
\FunctionTok{hist}\NormalTok{(results,}\AttributeTok{col=}\StringTok{\textquotesingle{}gray\textquotesingle{}}\NormalTok{,}\AttributeTok{main=}\StringTok{"Permutation Distribution"}\NormalTok{,}\AttributeTok{xlab=}\StringTok{"Simulated difference"}\NormalTok{)}
\FunctionTok{mean}\NormalTok{(results)}
\NormalTok{results }\OtherTok{\textless{}{-}} \FunctionTok{as.data.frame}\NormalTok{(results)}
\end{Highlighting}
\end{Shaded}

For the standard case of a permutation test, the null hypothesis is that
the two groups of interest come from the \emph{same distribution}, and
the alternative hypothesis is that the groups are not from the same
distribution. In the case of the matched-pairs permutation test, our
hypotheses are:

\[
\Large
H_0: \mu_{\text{diff}} = 0  
\]

In other words, the difference in the before and after values of points
is due to random chance

\[
\Large
H_a: \mu_{\text{diff}} \neq 0
\]

The permutation test computes 5,000 permutations of the 100,000 paired
points, where before/after values are switched. At each permutation, the
test statistic (the mean of all the paired differences) is computed.
This creates a roughly normal distribution of mean values. This works on
the assumption that if there truly is no difference in the distributions
of the before and after groups. Under the null assumption, the observed
test statistic (74.89 m, which we calculated from our original sample),
should fall somewhere within the ``null'' or permutation distribution.
To calculate our p-value, we take the proportion of all permutation
values that are equal to or more extreme than our test statistic.
Figure~\ref{fig-permHist} shows clearly that the test statistic obtained
from our sample (red line) is far more extreme than what can be expected
if there is no difference in the values of before/after groups.

\begin{figure}

\centering{

\pandocbounded{\includegraphics[keepaspectratio]{ProjectFigures/perm_hist.png}}

}

\caption{\label{fig-permHist}Visualization of permutation test for
statistically significant difference in the means of 2000 and 2024
distributions, p\textless0.0001}

\end{figure}%

\section{Results}\label{results}

here give summary statistics

i think would be good to include change in distance by land cover
classes, even ecoregions?

\begin{figure}

\centering{

\includegraphics[width=9.375in,height=\textheight,keepaspectratio]{ProjectFigures/facet_nlcd.png}

}

\caption{\label{fig-facet}Scatterplots of distance from the nearest road
in 2000 (x-axis) vs 2024(y-axis), separated by NLCD class. Color scale
representing proportion of total sampled points that became closer to
the nearest road.}

\end{figure}%

ridge plots could be good, simple bar chart showing mean change in
distance for each land cover type

ladder (?) plot showing change by land cover or ecoregion

``based on our analysis, X km of roads were constructed between 2000 and
2024, resulting in a measured shift of X with MEAN \_\_ and SD \_\_\_.
take from bootstrap sampling distribution?

\section{Tables}\label{tables}

\begin{figure}[H]

{\centering \pandocbounded{\includegraphics[keepaspectratio]{ProjectFigures/summary_stats1.png}}

}

\caption{Table showing summary statistics of each distribution, 2000 and
2024, in meters}

\end{figure}%

\section{Figures}\label{figures}

\begin{figure}[H]

{\centering \pandocbounded{\includegraphics[keepaspectratio]{ProjectFigures/whole_state_map.jpg}}

}

\caption{VA Map}

\end{figure}%

\begin{figure}[H]

{\centering \pandocbounded{\includegraphics[keepaspectratio]{ProjectFigures/plot-1284.png}}

}

\caption{(a). Histograms for two paired distributions, for the year 2000
and 2024, showing a left-shift from the measured values, indicating that
the measured points became closer to the nearest road, (b). Boxplot
showing the distributions of distance to the nearest road in each year,
(c). Density plot for a specific EPA level IV ecozone, Limestone Valleys
and Coves, indicating distances to the nearest road, (d). ridgeplot of
the measured values for 11 NLCD land use land cover classes}

\end{figure}%

\begin{figure}[H]

{\centering \pandocbounded{\includegraphics[keepaspectratio]{ProjectFigures/water_sampling.png}}

}

\caption{Example of points sampled along water features in Franklin
County, VA}

\end{figure}%

\begin{figure}[H]

{\centering \pandocbounded{\includegraphics[keepaspectratio]{ProjectFigures/map_with_labels.jpg}}

}

\caption{Map of Montgomery County, Virginia. Point color represents land
cover classification, and point size represents its distance to the
nearest road}

\end{figure}%

\subsection{Abbreviations}\label{abbreviations}

\begin{itemize}
\tightlist
\item
  \textbf{3DHP} 3D Hydrography Program
\item
  \textbf{EPA} Environmental Protection Agency
\item
  \textbf{GDAL} Geospatial Data Abstraction Library
\item
  \textbf{HUC} Hydrologic Unit Code
\item
  \textbf{NLCD} National Land Cover Database
\item
  \textbf{TIGER} Topologically Integrated Geographic Encoding and
  Referencing
\item
  \textbf{USGS} United States Geoglogical Survey
\end{itemize}

\section*{References}\label{references}
\addcontentsline{toc}{section}{References}

\phantomsection\label{refs}
\begin{CSLReferences}{1}{0}
\bibitem[\citeproctext]{ref-usgs}
{``Annual NLCD Collection 1 Science Products.''} 2024. US Geological
Survey (USGS).

\bibitem[\citeproctext]{ref-dixon.2001}
Dixon, Philip M. 2001. {``Bootstrap Resampling.''} \emph{Encyclopedia of
Environmetrics}, October.
\url{https://doi.org/10.1002/9780470057339.VAB028}.

\bibitem[\citeproctext]{ref-blanka.etal.2020}
Gál, Blanka, András Weiperth, János Farkas, and Dénes Schmera. 2020.
{``The Effects of Road Crossings on Stream Macro-Invertebrate
Diversity.''} \emph{Biodiversity and Conservation} 29 (March): 729--45.
\url{https://doi.org/10.1007/S10531-019-01907-4}.

\bibitem[\citeproctext]{ref-geopandas}
Jordahl, Kelsey, Joris Van den Bossche, Martin Fleischmann, Jacob
Wasserman, James McBride, Jeffrey Gerard, Jeff Tratner, et al. 2020.
{``Geopandas/Geopandas: V0.8.1.''} Zenodo.
\url{https://doi.org/10.5281/zenodo.3946761}.

\bibitem[\citeproctext]{ref-matched.perm}
{``Permutation Test for Matched Pairs Data.''} n.d.
\url{https://people.hsc.edu/faculty-staff/blins/classes/spring19/math222/Examples/MatchedPairsPermutationTest.html}.

\bibitem[\citeproctext]{ref-pygris}
{``Pygris.''} n.d. \url{https://walker-data.com/pygris/}.

\bibitem[\citeproctext]{ref-beautifulsoup}
Richardson, Leonard. 2007. {``Beautiful Soup Documentation.''}
\emph{April}.

\bibitem[\citeproctext]{ref-riitters.etal.2003}
Riitters, Kurt H, and James D Wickham. 2003. {``How Far to the Nearest
Road?''} \emph{Ecology and the Environment}. Vol. 1.

\bibitem[\citeproctext]{ref-szekely.gabor.2004}
Szekely, Gabor J, Maria L Rizzo, and Gábor J Székely. 2004. {``Testing
for Equal Distributions in High Dimension.''}
\url{https://www.researchgate.net/publication/228918499}.

\bibitem[\citeproctext]{ref-welch.1990}
Welch, William J. 1990. {``Construction of Permutation Tests.''}
\emph{Journal of the American Statistical Association} 85: 693--98.
\url{https://doi.org/10.1080/01621459.1990.10474929}.

\end{CSLReferences}




\end{document}
