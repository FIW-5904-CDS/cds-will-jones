% Options for packages loaded elsewhere
% Options for packages loaded elsewhere
\PassOptionsToPackage{unicode}{hyperref}
\PassOptionsToPackage{hyphens}{url}
\PassOptionsToPackage{dvipsnames,svgnames,x11names}{xcolor}
%
\documentclass[
  letterpaper,
  DIV=11,
  numbers=noendperiod]{scrartcl}
\usepackage{xcolor}
\usepackage{amsmath,amssymb}
\setcounter{secnumdepth}{5}
\usepackage{iftex}
\ifPDFTeX
  \usepackage[T1]{fontenc}
  \usepackage[utf8]{inputenc}
  \usepackage{textcomp} % provide euro and other symbols
\else % if luatex or xetex
  \usepackage{unicode-math} % this also loads fontspec
  \defaultfontfeatures{Scale=MatchLowercase}
  \defaultfontfeatures[\rmfamily]{Ligatures=TeX,Scale=1}
\fi
\usepackage{lmodern}
\ifPDFTeX\else
  % xetex/luatex font selection
\fi
% Use upquote if available, for straight quotes in verbatim environments
\IfFileExists{upquote.sty}{\usepackage{upquote}}{}
\IfFileExists{microtype.sty}{% use microtype if available
  \usepackage[]{microtype}
  \UseMicrotypeSet[protrusion]{basicmath} % disable protrusion for tt fonts
}{}
\makeatletter
\@ifundefined{KOMAClassName}{% if non-KOMA class
  \IfFileExists{parskip.sty}{%
    \usepackage{parskip}
  }{% else
    \setlength{\parindent}{0pt}
    \setlength{\parskip}{6pt plus 2pt minus 1pt}}
}{% if KOMA class
  \KOMAoptions{parskip=half}}
\makeatother
% Make \paragraph and \subparagraph free-standing
\makeatletter
\ifx\paragraph\undefined\else
  \let\oldparagraph\paragraph
  \renewcommand{\paragraph}{
    \@ifstar
      \xxxParagraphStar
      \xxxParagraphNoStar
  }
  \newcommand{\xxxParagraphStar}[1]{\oldparagraph*{#1}\mbox{}}
  \newcommand{\xxxParagraphNoStar}[1]{\oldparagraph{#1}\mbox{}}
\fi
\ifx\subparagraph\undefined\else
  \let\oldsubparagraph\subparagraph
  \renewcommand{\subparagraph}{
    \@ifstar
      \xxxSubParagraphStar
      \xxxSubParagraphNoStar
  }
  \newcommand{\xxxSubParagraphStar}[1]{\oldsubparagraph*{#1}\mbox{}}
  \newcommand{\xxxSubParagraphNoStar}[1]{\oldsubparagraph{#1}\mbox{}}
\fi
\makeatother

\usepackage{color}
\usepackage{fancyvrb}
\newcommand{\VerbBar}{|}
\newcommand{\VERB}{\Verb[commandchars=\\\{\}]}
\DefineVerbatimEnvironment{Highlighting}{Verbatim}{commandchars=\\\{\}}
% Add ',fontsize=\small' for more characters per line
\usepackage{framed}
\definecolor{shadecolor}{RGB}{241,243,245}
\newenvironment{Shaded}{\begin{snugshade}}{\end{snugshade}}
\newcommand{\AlertTok}[1]{\textcolor[rgb]{0.68,0.00,0.00}{#1}}
\newcommand{\AnnotationTok}[1]{\textcolor[rgb]{0.37,0.37,0.37}{#1}}
\newcommand{\AttributeTok}[1]{\textcolor[rgb]{0.40,0.45,0.13}{#1}}
\newcommand{\BaseNTok}[1]{\textcolor[rgb]{0.68,0.00,0.00}{#1}}
\newcommand{\BuiltInTok}[1]{\textcolor[rgb]{0.00,0.23,0.31}{#1}}
\newcommand{\CharTok}[1]{\textcolor[rgb]{0.13,0.47,0.30}{#1}}
\newcommand{\CommentTok}[1]{\textcolor[rgb]{0.37,0.37,0.37}{#1}}
\newcommand{\CommentVarTok}[1]{\textcolor[rgb]{0.37,0.37,0.37}{\textit{#1}}}
\newcommand{\ConstantTok}[1]{\textcolor[rgb]{0.56,0.35,0.01}{#1}}
\newcommand{\ControlFlowTok}[1]{\textcolor[rgb]{0.00,0.23,0.31}{\textbf{#1}}}
\newcommand{\DataTypeTok}[1]{\textcolor[rgb]{0.68,0.00,0.00}{#1}}
\newcommand{\DecValTok}[1]{\textcolor[rgb]{0.68,0.00,0.00}{#1}}
\newcommand{\DocumentationTok}[1]{\textcolor[rgb]{0.37,0.37,0.37}{\textit{#1}}}
\newcommand{\ErrorTok}[1]{\textcolor[rgb]{0.68,0.00,0.00}{#1}}
\newcommand{\ExtensionTok}[1]{\textcolor[rgb]{0.00,0.23,0.31}{#1}}
\newcommand{\FloatTok}[1]{\textcolor[rgb]{0.68,0.00,0.00}{#1}}
\newcommand{\FunctionTok}[1]{\textcolor[rgb]{0.28,0.35,0.67}{#1}}
\newcommand{\ImportTok}[1]{\textcolor[rgb]{0.00,0.46,0.62}{#1}}
\newcommand{\InformationTok}[1]{\textcolor[rgb]{0.37,0.37,0.37}{#1}}
\newcommand{\KeywordTok}[1]{\textcolor[rgb]{0.00,0.23,0.31}{\textbf{#1}}}
\newcommand{\NormalTok}[1]{\textcolor[rgb]{0.00,0.23,0.31}{#1}}
\newcommand{\OperatorTok}[1]{\textcolor[rgb]{0.37,0.37,0.37}{#1}}
\newcommand{\OtherTok}[1]{\textcolor[rgb]{0.00,0.23,0.31}{#1}}
\newcommand{\PreprocessorTok}[1]{\textcolor[rgb]{0.68,0.00,0.00}{#1}}
\newcommand{\RegionMarkerTok}[1]{\textcolor[rgb]{0.00,0.23,0.31}{#1}}
\newcommand{\SpecialCharTok}[1]{\textcolor[rgb]{0.37,0.37,0.37}{#1}}
\newcommand{\SpecialStringTok}[1]{\textcolor[rgb]{0.13,0.47,0.30}{#1}}
\newcommand{\StringTok}[1]{\textcolor[rgb]{0.13,0.47,0.30}{#1}}
\newcommand{\VariableTok}[1]{\textcolor[rgb]{0.07,0.07,0.07}{#1}}
\newcommand{\VerbatimStringTok}[1]{\textcolor[rgb]{0.13,0.47,0.30}{#1}}
\newcommand{\WarningTok}[1]{\textcolor[rgb]{0.37,0.37,0.37}{\textit{#1}}}

\usepackage{longtable,booktabs,array}
\usepackage{calc} % for calculating minipage widths
% Correct order of tables after \paragraph or \subparagraph
\usepackage{etoolbox}
\makeatletter
\patchcmd\longtable{\par}{\if@noskipsec\mbox{}\fi\par}{}{}
\makeatother
% Allow footnotes in longtable head/foot
\IfFileExists{footnotehyper.sty}{\usepackage{footnotehyper}}{\usepackage{footnote}}
\makesavenoteenv{longtable}
\usepackage{graphicx}
\makeatletter
\newsavebox\pandoc@box
\newcommand*\pandocbounded[1]{% scales image to fit in text height/width
  \sbox\pandoc@box{#1}%
  \Gscale@div\@tempa{\textheight}{\dimexpr\ht\pandoc@box+\dp\pandoc@box\relax}%
  \Gscale@div\@tempb{\linewidth}{\wd\pandoc@box}%
  \ifdim\@tempb\p@<\@tempa\p@\let\@tempa\@tempb\fi% select the smaller of both
  \ifdim\@tempa\p@<\p@\scalebox{\@tempa}{\usebox\pandoc@box}%
  \else\usebox{\pandoc@box}%
  \fi%
}
% Set default figure placement to htbp
\def\fps@figure{htbp}
\makeatother


% definitions for citeproc citations
\NewDocumentCommand\citeproctext{}{}
\NewDocumentCommand\citeproc{mm}{%
  \begingroup\def\citeproctext{#2}\cite{#1}\endgroup}
\makeatletter
 % allow citations to break across lines
 \let\@cite@ofmt\@firstofone
 % avoid brackets around text for \cite:
 \def\@biblabel#1{}
 \def\@cite#1#2{{#1\if@tempswa , #2\fi}}
\makeatother
\newlength{\cslhangindent}
\setlength{\cslhangindent}{1.5em}
\newlength{\csllabelwidth}
\setlength{\csllabelwidth}{3em}
\newenvironment{CSLReferences}[2] % #1 hanging-indent, #2 entry-spacing
 {\begin{list}{}{%
  \setlength{\itemindent}{0pt}
  \setlength{\leftmargin}{0pt}
  \setlength{\parsep}{0pt}
  % turn on hanging indent if param 1 is 1
  \ifodd #1
   \setlength{\leftmargin}{\cslhangindent}
   \setlength{\itemindent}{-1\cslhangindent}
  \fi
  % set entry spacing
  \setlength{\itemsep}{#2\baselineskip}}}
 {\end{list}}
\usepackage{calc}
\newcommand{\CSLBlock}[1]{\hfill\break\parbox[t]{\linewidth}{\strut\ignorespaces#1\strut}}
\newcommand{\CSLLeftMargin}[1]{\parbox[t]{\csllabelwidth}{\strut#1\strut}}
\newcommand{\CSLRightInline}[1]{\parbox[t]{\linewidth - \csllabelwidth}{\strut#1\strut}}
\newcommand{\CSLIndent}[1]{\hspace{\cslhangindent}#1}



\setlength{\emergencystretch}{3em} % prevent overfull lines

\providecommand{\tightlist}{%
  \setlength{\itemsep}{0pt}\setlength{\parskip}{0pt}}



 


\KOMAoption{captions}{tableheading}
\makeatletter
\@ifpackageloaded{caption}{}{\usepackage{caption}}
\AtBeginDocument{%
\ifdefined\contentsname
  \renewcommand*\contentsname{Table of contents}
\else
  \newcommand\contentsname{Table of contents}
\fi
\ifdefined\listfigurename
  \renewcommand*\listfigurename{List of Figures}
\else
  \newcommand\listfigurename{List of Figures}
\fi
\ifdefined\listtablename
  \renewcommand*\listtablename{List of Tables}
\else
  \newcommand\listtablename{List of Tables}
\fi
\ifdefined\figurename
  \renewcommand*\figurename{Figure}
\else
  \newcommand\figurename{Figure}
\fi
\ifdefined\tablename
  \renewcommand*\tablename{Table}
\else
  \newcommand\tablename{Table}
\fi
}
\@ifpackageloaded{float}{}{\usepackage{float}}
\floatstyle{ruled}
\@ifundefined{c@chapter}{\newfloat{codelisting}{h}{lop}}{\newfloat{codelisting}{h}{lop}[chapter]}
\floatname{codelisting}{Listing}
\newcommand*\listoflistings{\listof{codelisting}{List of Listings}}
\makeatother
\makeatletter
\makeatother
\makeatletter
\@ifpackageloaded{caption}{}{\usepackage{caption}}
\@ifpackageloaded{subcaption}{}{\usepackage{subcaption}}
\makeatother
\usepackage{bookmark}
\IfFileExists{xurl.sty}{\usepackage{xurl}}{} % add URL line breaks if available
\urlstyle{same}
\hypersetup{
  pdftitle={Geospatial Analysis of a Changing Road Network},
  pdfauthor={Will jones; Ben Jantzen; Yang Shao},
  colorlinks=true,
  linkcolor={blue},
  filecolor={Maroon},
  citecolor={Blue},
  urlcolor={Blue},
  pdfcreator={LaTeX via pandoc}}


\title{Geospatial Analysis of a Changing Road Network}
\author{Will jones \and Ben Jantzen \and Yang Shao}
\date{2025-11-19}
\begin{document}
\maketitle

\renewcommand*\contentsname{Table of contents}
{
\hypersetup{linkcolor=}
\setcounter{tocdepth}{4}
\tableofcontents
}

\section{Abstract}\label{abstract}

This study investigates the expansion of the road network in the
contiguous United States from 2000 to 2024 and its implications for
landscape ecology. Using GIS tools and statistical analyses, we examine
changes in the proximity of terrestrial and aquatic features to the
nearest road. The analysis employs permutation testing and bootstrap
resampling to assess the statistical significance of shifts in road
proximity over time. The findings highlight the growing influence of
road infrastructure on both terrestrial and aquatic ecosystems, with
potential implications for habitat fragmentation, water quality, and
species connectivity.

\section{Introduction}\label{introduction}

The 2003 study ``How Far to the Nearest Road ?'' published in
\emph{Frontiers in Ecology and The Environment} studied the road network
of the United States (\citeproc{ref-riitters.etal.2003}{Riitters and
Wickham 2003}). Specifically, it split the contiguous US into 30x30 m
grid cells and classified each as being a ``road'' or ``non-road'' cell,
based on if any part of the cell intersects a TIGER/line road feature.
Then, a \textbf{binned classification} was used to measure the distance
of every cell to its nearest road cell.

Our aim is to expand on this methodology using GIS and statistical
tests. Specifically, our objective is to answer the following questions:

\begin{enumerate}
\def\labelenumi{\arabic{enumi}.}
\tightlist
\item
  By randomly sampling points and calculating straight-line (Euclidean)
  distances, can we measure a more accurate continuous distribution of
  distance to the nearest road?
\item
  Using archived Census Bureau TIGER/Line geographic data, can we
  measure the average change in distance to the nearest road for the
  contiguous US?
\item
  \textbf{What water bodies have experienced the greatest change and
  become more vulnerable to ecological effects of road construction and
  operation?}
\end{enumerate}

The role of road networks related to habitat fragmentation, and its
subsequent effects on plant and animal life have been studied
extensively (\citeproc{ref-saura}{Saura and Rubio 2010};
\citeproc{ref-Forman}{Forman and Alexander 1998};
\citeproc{ref-small.mammals}{McGregor, Bender, and Fahrig 2008}). This
is a global topic of interest in landscape ecology, with studies of road
impacts across various ecozones in Sweden
(\citeproc{ref-Karlson}{Karlson and Mörtberg 2015}), landscape
ecological risk (LER) related to roads in the Central Himalaya
(\citeproc{ref-mann}{Mann et al. 2021}), giant anteater mortality in
Brazil (\citeproc{ref-pinto}{Pinto et al. 2018}), and many many more.

Functional connectivity, a foundational concept in landscape ecology
refers to the degree to which organisms can perform certain biological
processes across their landscape. Such processes include dispersal of
seeds, migration related to breeding, and genetic exchange. ``By
severing habitat connections, roads create risk for animals, especially
slow-moving reptiles like the wood turtle,'' according to the
Conservation Planning in the Hudson River Estuary Watershed project at
Cornell University (\citeproc{ref-cornell}{{``Connectivity Planning
\textbar{} Conservation Planning in the Hudson River Estuary
Watershed,''} n.d.})

Within the scope of animals vulnerable to the impacts of road effects,
native stream invertebrate species may be especially at risk. In fact,
Gál, et al.~found that road crossings over streams (bridges) had
negative effects on the biodiversity of native invertebrates in Hungary,
including species richness, abundance, and prevalence of protected
species (\citeproc{ref-blanka.etal.2020}{Gál et al. 2020}). Despite
this, road expansion has continued at a steady rate in the United
States, with little long-term monitoring of the growing impact on
landscape biodiversity. Between 1980 and 2011, 183,000 miles of roads
have been built, an average of 6,500 miles of road per year.
Additionally, 32,264 bridges were constructed between 1992 and 2010,
bringing the total as of 2010 to 604,460
(\citeproc{ref-highway}{{``Office of Highway Policy Information - Policy
\textbar{} Federal Highway Administration,''} n.d.}). Of course, this
period encompasses the time since the last contiguous United
States-level measurement of distances to the nearest road in 2003
(\citeproc{ref-riitters.etal.2003}{Riitters and Wickham 2003}). Using
GIS methods and a statistical approach based on the energy distance
(\citeproc{ref-szekely.gabor.2004}{Szekely and Rizzo 2004}), we may
examine the expansion of the United States road network for future
ecological applications. Additionally, through measurement of a paired
continuous distribution of distances (2000 and 2024), we may observe the
magnitude of change in distances, both at the national scale, and at
relevant sub-levels, including watershed, EPA ecoregion, NLCD landcover
class, and even individual stream level. (*\emph{Note: for this analysis
I am using a permutation test, not the energy distance method}).

\section{Methods}\label{methods}

\subsection{Data Acquisition and
Preprocessing}\label{data-acquisition-and-preprocessing}

Six primary spatial datasets were used in our analysis. The 2023 USGS
National Land Cover Database (NLCD), containing 11 distinct land cover
classes, was used to extract surface type at each of our sampled points
(\citeproc{ref-usgs}{{``Annual NLCD Collection 1 Science Products''}
2024}). Likewise, attribute data for each sampled point was extracted
using the EPA's ecoregion level IV map, and watershed membership was
extracted using the USGS 3D Hydrography Program (3DHP) data set. 3DHP is
a unified lidar-derived data product containing vector data for streams,
lakes, ponds, catchments, and other hydrologic features (USGS, 2025).

US Census Bureau's TIGER/Line vector files were used as the road
networks in this project. The most recently available 2024 road network
data for the conterminous United States was downloaded in bulk using the
\texttt{roads()} function from the Python package \texttt{pygris}
(\citeproc{ref-pygris}{{``Pygris,''} n.d.}).

Comparable data for the year 2000 was downloaded from the archived
Census FTP site (\url{https://www2.census.gov/geo/tiger/tiger2k/),} and
a multi-step data preprocessing workflow was performed before the data
was implemented in the analysis workflow. First, \texttt{beautifulsoup4}
was used to download the vector dataset in RT1 format, the historical
file format for census geographic data
(\citeproc{ref-beautifulsoup}{Richardson 2007}). Using GDAL command-line
utilities, files were converted to modern shapefile format, then merged
into a single dataset using \texttt{geopandas} in python
(\citeproc{ref-geopandas}{Jordahl et al. 2020}). Finally, feature
correction was performed using geometry-snapping and densification
processing in QGIS 3.36 ''Maidenhead.'' To rectify data quality issues,
namely feature accuracy and spatial mismatch between existing roads in
the 2000 and 2024 datasets, the historical vector dataset was densified
at regular 25m intervals, then snapped to the nearest matching feature
in the current roads dataset. This was done to avoid erroneous
measurement differences in distance to the nearest road for each year
that may be incorrectly attributed to removal or addition of a new road
feature.

\subsection{Sampling Scheme and
Measurement}\label{sampling-scheme-and-measurement}

\subsubsection{Terrestrial Points}\label{terrestrial-points}

Using the R package sf, 1,000,000 points were randomly sampled across
our study area (Virginia, USA). Points were sampled uniformly across
space, meaning that everywhere on the continuous surface has equal
likelihood of being sampled (Fotheringham and Rogerson 2008). Next, the
nearest feature from each road network (2000 and 2024) was identified
for each randomly sampled point, and the Euclidean distance between the
two geometries was measured. The result was two continuous
distributions, each containing a measurement at the same point. This is
a case of paired data, analogous to `before and after' measurements.

\subsubsection{Water Points}\label{water-points}

To perform the corresponding process for water features (lakes, ponds,
streams, creeks), a slightly different sampling method was used. It is
our goal to randomly select points along linear water features and along
the \emph{edges} of waterbodies (polygons). We do not want to sample
points within ponds or lakes, because the measurement of interest is the
distance between the least central points of each areal feature (i.e.,
the ``shore'') and the nearest road feature. This effectively captures
where ecological interactions such as runoff into ponds/lakes will
occur. To do this, the layers \texttt{hydro\_3dhp\_all\_flowline} and
\texttt{hydro\_3dhp\_all\_waterbody} were extracted from the USGS 3DHP
geodatabase, representing flowlines and waterbodies, respectively. Due
to the structure of the 3DHP features, \texttt{flowlines} are often
drawn through the waterbody that they feed into. To navigate this, and
avoid sampling within water bodies, the \texttt{sf} package function
\texttt{st\_difference} was used to eliminate these overlapping
flowlines from the features that could be sampled on. Lastly, lake and
pond boundaries were merged with stream and river features into a single
linear feature layer. When using \texttt{st\_sample}, this ensures that
the points are uniformly randomly sampled, meaning that any point in our
dataset has equal probability of being sampled. (\emph{Note: the reason
analysis of these water points has not been accomplished is because we
have not been able to generate a sufficiently large sample. This is
because the geometric operations for a huge set of points are
computationally intensive. Our future plan is to do this processing on
Virginia Tech's Advanced Research Computing resources}).

\begin{figure}

\begin{minipage}{0.50\linewidth}

\centering{

\pandocbounded{\includegraphics[keepaspectratio]{ProjectFigures/wrong_water_pts.png}}

}

\subcaption{\label{fig-wrong}Sampling on water features before removing
overlapping flowlines}

\end{minipage}%
%
\begin{minipage}{0.50\linewidth}

\centering{

\pandocbounded{\includegraphics[keepaspectratio]{ProjectFigures/good_water_pts.png}}

}

\subcaption{\label{fig-right}Sampling on water features after removing
overlaping flowlines}

\end{minipage}%

\end{figure}%

\subsection{Statistical Analysis}\label{statistical-analysis}

For the purposes of this class project, 100,000 of the
\textasciitilde1e6 points were chosen randomly for analysis in order to
make local processing possible.

To compare the shift in distributions from 2000 to 2024, a permutation
test for matched pairs was used (\citeproc{ref-welch.1990}{Welch 1990}),
along with bootstrap resampling (\citeproc{ref-dixon.2001}{Dixon 2001})
to construct a confidence interval for the test statistic of interest,
the mean of the differences between the paired points. The mean of the
paired differences was chosen as the test statistic for analysis because
the data set is dominated by points whose values did not change (i.e.,
difference is 0). This characteristic of the distribution, along with
the fact that the distribution of differences is not symmetric around
the mean (see Figure~\ref{fig-qq}), is why we did not use the Wilcoxon
Signed Rank test.

\begin{figure}

\centering{

\pandocbounded{\includegraphics[keepaspectratio]{ProjectFigures/differences_qq.png}}

}

\caption{\label{fig-qq}}

\end{figure}%

We used bootstrapping (\citeproc{ref-dixon.2001}{Dixon 2001}) to
construct a sampling distribution and confidence interval for the mean
of paired differences. The observed mean of the differences between our
paired data is 74.89 meters. After bootstrap resampling, we have an
approximated sampling distribution with a mean of 74.87 meters (95\% CI:
(73.19, 76.60)) and a standard error of 0.87 meters.

\pandocbounded{\includegraphics[keepaspectratio]{ProjectFigures/bootstrap_hist.png}}

\subsubsection{Permutation Test}\label{permutation-test}

Then we performed a permutation test for paired data
(\citeproc{ref-matched.perm}{{``Permutation Test for Matched Pairs
Data,''} n.d.}) to test for significance in our results:

\begin{Shaded}
\begin{Highlighting}[]
\NormalTok{results }\OtherTok{=} \FunctionTok{c}\NormalTok{()}
\ControlFlowTok{for}\NormalTok{ (i }\ControlFlowTok{in} \DecValTok{1}\SpecialCharTok{:}\DecValTok{5000}\NormalTok{) \{}
\NormalTok{  permutedData }\OtherTok{=} \FunctionTok{sample}\NormalTok{(}\FunctionTok{c}\NormalTok{(}\DecValTok{1}\NormalTok{,}\SpecialCharTok{{-}}\DecValTok{1}\NormalTok{),}\DecValTok{100000}\NormalTok{,}\AttributeTok{replace=}\NormalTok{T)}\SpecialCharTok{*}\NormalTok{data}\SpecialCharTok{$}\NormalTok{difference}
\NormalTok{  results }\OtherTok{=} \FunctionTok{c}\NormalTok{(}\FunctionTok{mean}\NormalTok{(permutedData),results)}
\NormalTok{\}}

\NormalTok{results }\OtherTok{\textless{}{-}} \FunctionTok{as.data.frame}\NormalTok{(results)}
\end{Highlighting}
\end{Shaded}

The above code generates 5,000 values stored in the vector
\texttt{results}. \textbf{Each value is a simulated test statistic,
creating an approximated sampling distribution.}

For the standard case of a permutation test, the null hypothesis is that
the two groups of interest come from the \emph{same distribution}, and
the alternative hypothesis is that the groups are not from the same
distribution. In the case of the matched-pairs permutation test, our
hypotheses are:

\[
\Large
H_0: \mu_{\text{diff}} = 0  
\]

In other words, the difference in the before and after values of points
is due to random chance / there is no true difference in the average
distance between points.

\[
\Large
H_a: \mu_{\text{diff}} \neq 0
\]

The permutation test computes 5,000 permutations of the 100,000 paired
points, where the signs of the differences of before/after values are
randomly flipped. At each permutation, the test statistic (the mean of
all the paired differences) is computed. This creates a (usually)
roughly normal distribution of mean values under the null. This works on
the assumption that there truly is no difference in the distributions of
the before and after groups. Under the null assumption, the observed
test statistic (74.89 m, which we calculated from our original sample),
should fall somewhere within the ``null'' or permutation distribution.
To calculate our p-value, we take the proportion of all permutation
values that are equal to or more extreme than our test statistic.
Figure~\ref{fig-permHist} shows clearly that the test statistic obtained
from our sample (red line) is far more extreme than what can be expected
if there is no difference in the values of before/after groups.

\begin{figure}

\centering{

\pandocbounded{\includegraphics[keepaspectratio]{ProjectFigures/perm_hist.png}}

}

\caption{\label{fig-permHist}Visualization of permutation test for
statistically significant difference in the means of 2000 and 2024
distributions, p\textless0.0002}

\end{figure}%

\section{Results}\label{results}

The results of our permutation test are indeed significant, and support
the alternative hypothesis: the average change in distance to the
nearest road \textbf{has} significantly \textbf{decreased}. To the best
of our knowledge, according to our bootstrap resampling, the true mean
change in distance is approximately 74.87 +/- 0.87 meters, with a 95\%
CI of 73.19 m, 76.60 m.

Future work is needed to apply this methodology to freshwater features.
This is actually the most important part of our analysis, since our goal
is to study the effects of roads on native invertebrate populations.
Below, Figure~\ref{fig-nlcd-facet} and Figure~\ref{fig-eco-facet} show
visually the impact of road construction in Virginia, separated by land
cover class and selected EPA ecoregions. This is a rough way to assess
what areas the study area have been most ``effected,'' or in other
words, which areas have had the most points become \textbf{closer} to
the nearest road. In the figures below, all the points below the dotted
1:1 line (indicating no change), have become closer to the nearest road,
which increases the chance of negative ecological effects, such as
pollution from runoff, bank erosion, and increased risk of animal
collision. It is worth noting that all scatterplots show a general trend
of having \textbf{more points under the no change line than over the no
change line.} This indicates widespread encroachment of road features
across all land covers and ecoregions. However, in all cases, the
majority of points fall \textbf{exactly} \textbf{on the no change line.}
What this means is that the vast majority of points have not changed.
This added certain complications in our statistical analysis. For
example, the difference in the medians of the distributions is non-zero,
however, the median of the differences between the paired points
\textbf{is} 0. Since this is a pairwise analysis with unique data
structure, we must be careful when trying to draw inferences from our
data.

\begin{figure}

\begin{minipage}{0.50\linewidth}

\begin{figure}[H]

\centering{

\includegraphics[width=9.375in,height=\textheight,keepaspectratio]{ProjectFigures/facet_nlcd.png}

}

\caption{\label{fig-nlcd-facet}Scatterplots of distance from the nearest
road in 2000 (x-axis) vs 2024(y-axis), separated by NLCD class. Color
scale representing proportion of total sampled points that became closer
to the nearest road.}

\end{figure}%

\end{minipage}%
%
\begin{minipage}{0.50\linewidth}

\begin{figure}[H]

\centering{

\pandocbounded{\includegraphics[keepaspectratio]{ProjectFigures/eco_facet.png}}

}

\caption{\label{fig-eco-facet}Scatterplots of distance from the nearest
road in 2000 (x-axis) vs 2024(y-axis), separated by selected EPA
ecoregion. Color scale representing proportion of total sampled points
that became closer to the nearest road.}

\end{figure}%

\end{minipage}%

\end{figure}%

\section{Tables}\label{tables}

\begin{figure}[H]

{\centering \pandocbounded{\includegraphics[keepaspectratio]{ProjectFigures/summary_stats1.png}}

}

\caption{Table showing summary statistics of each distribution, 2000 and
2024, in meters}

\end{figure}%

\section{Figures}\label{figures}

\begin{figure}

\centering{

\pandocbounded{\includegraphics[keepaspectratio]{ProjectFigures/map_with_labels.jpg}}

}

\caption{\label{fig-mont}Map of Montgomery County, Virginia. Point color
represents land cover classification, and point size represents its
distance to the nearest road}

\end{figure}%

\begin{figure}[H]

{\centering \pandocbounded{\includegraphics[keepaspectratio]{ProjectFigures/whole_state_map.jpg}}

}

\caption{Map of data across Virginia. Symbology is the same as in
Figure~\ref{fig-mont}.}

\end{figure}%

\begin{figure}

\centering{

\pandocbounded{\includegraphics[keepaspectratio]{ProjectFigures/plot-1284.png}}

}

\caption{\label{fig-stack}(a). Histograms for two paired distributions,
for the year 2000 and 2024, showing a left-shift from the measured
values, indicating that the measured points became closer to the nearest
road, (b). Boxplot showing the distributions of distance to the nearest
road in each year, (c). Density plot for a specific EPA level IV
ecozone, Limestone Valleys and Coves, indicating distances to the
nearest road, (d). ridgeplot of the measured values for 11 NLCD land use
land cover classes}

\end{figure}%

\begin{figure}[H]

{\centering \pandocbounded{\includegraphics[keepaspectratio]{ProjectFigures/water_sampling.png}}

}

\caption{Example of points sampled along water features in Franklin
County, VA}

\end{figure}%

\subsection{Abbreviations}\label{abbreviations}

\begin{itemize}
\tightlist
\item
  \textbf{3DHP} 3D Hydrography Program
\item
  \textbf{EPA} Environmental Protection Agency
\item
  \textbf{GDAL} Geospatial Data Abstraction Library
\item
  \textbf{HUC} Hydrologic Unit Code
\item
  \textbf{NLCD} National Land Cover Database
\item
  \textbf{TIGER} Topologically Integrated Geographic Encoding and
  Referencing
\item
  \textbf{USGS} United States Geoglogical Survey
\end{itemize}

\section*{References}\label{references}
\addcontentsline{toc}{section}{References}

\phantomsection\label{refs}
\begin{CSLReferences}{1}{0}
\bibitem[\citeproctext]{ref-usgs}
{``Annual NLCD Collection 1 Science Products.''} 2024. US Geological
Survey (USGS).

\bibitem[\citeproctext]{ref-cornell}
{``Connectivity Planning \textbar{} Conservation Planning in the Hudson
River Estuary Watershed.''} n.d.
\url{https://hudson.dnr.cals.cornell.edu/conservation-planning/inventory-and-planning/connectivity-planning}.

\bibitem[\citeproctext]{ref-dixon.2001}
Dixon, Philip M. 2001. {``Bootstrap Resampling.''} \emph{Encyclopedia of
Environmetrics}, October.
\url{https://doi.org/10.1002/9780470057339.VAB028}.

\bibitem[\citeproctext]{ref-Forman}
Forman, Richard T. T., and Lauren E. Alexander. 1998. {``Roads and Their
Major Ecological Effects.''} \emph{Annual Review of Ecology and
Systematics} 29: 207--31.
\url{https://doi.org/10.1146/ANNUREV.ECOLSYS.29.1.207}.

\bibitem[\citeproctext]{ref-blanka.etal.2020}
Gál, Blanka, András Weiperth, János Farkas, and Dénes Schmera. 2020.
{``The Effects of Road Crossings on Stream Macro-Invertebrate
Diversity.''} \emph{Biodiversity and Conservation} 29 (March): 729--45.
\url{https://doi.org/10.1007/S10531-019-01907-4}.

\bibitem[\citeproctext]{ref-geopandas}
Jordahl, Kelsey, Joris Van den Bossche, Martin Fleischmann, Jacob
Wasserman, James McBride, Jeffrey Gerard, Jeff Tratner, et al. 2020.
{``Geopandas/Geopandas: V0.8.1.''} Zenodo.
\url{https://doi.org/10.5281/zenodo.3946761}.

\bibitem[\citeproctext]{ref-Karlson}
Karlson, Mårten, and Ulla Mörtberg. 2015. {``A Spatial Ecological
Assessment of Fragmentation and Disturbance Effects of the Swedish Road
Network.''} \emph{Landscape and Urban Planning} 134 (February): 53--65.
\url{https://doi.org/10.1016/J.LANDURBPLAN.2014.10.009}.

\bibitem[\citeproctext]{ref-mann}
Mann, Deepika, Mangalasseril Mohammad Anees, Shalini Rankavat, and Pawan
Kumar Joshi. 2021. {``Spatio-Temporal Variations in Landscape Ecological
Risk Related to Road Network in the Central Himalaya.''} \emph{Human and
Ecological Risk Assessment} 27: 289--306.
\url{https://doi.org/10.1080/10807039.2019.1710693}.

\bibitem[\citeproctext]{ref-small.mammals}
McGregor, Rachelle L., Darren J. Bender, and Lenore Fahrig. 2008. {``Do
Small Mammals Avoid Roads Because of the Traffic?''} \emph{Journal of
Applied Ecology} 45 (February): 117--23.
\url{https://doi.org/10.1111/J.1365-2664.2007.01403.X}.

\bibitem[\citeproctext]{ref-highway}
{``Office of Highway Policy Information - Policy \textbar{} Federal
Highway Administration.''} n.d.
\url{https://www.fhwa.dot.gov/policyinformation/pubs/hf/pl11028/chapter1.cfm}.

\bibitem[\citeproctext]{ref-matched.perm}
{``Permutation Test for Matched Pairs Data.''} n.d.
\url{https://people.hsc.edu/faculty-staff/blins/classes/spring19/math222/Examples/MatchedPairsPermutationTest.html}.

\bibitem[\citeproctext]{ref-pinto}
Pinto, Fernando A. S., Alex Bager, Anthony P. Clevenger, and Clara
Grilo. 2018. {``Giant Anteater (Myrmecophaga Tridactyla) Conservation in
Brazil: Analysing the Relative Effects of Fragmentation and Mortality
Due to Roads.''} \emph{Biological Conservation} 228 (December): 148--57.
\url{https://doi.org/10.1016/J.BIOCON.2018.10.023}.

\bibitem[\citeproctext]{ref-pygris}
{``Pygris.''} n.d. \url{https://walker-data.com/pygris/}.

\bibitem[\citeproctext]{ref-beautifulsoup}
Richardson, Leonard. 2007. {``Beautiful Soup Documentation.''}
\emph{April}.

\bibitem[\citeproctext]{ref-riitters.etal.2003}
Riitters, Kurt H, and James D Wickham. 2003. {``How Far to the Nearest
Road?''} \emph{Ecology and the Environment}. Vol. 1.

\bibitem[\citeproctext]{ref-saura}
Saura, Santiago, and Lidón Rubio. 2010. {``A Common Currency for the
Different Ways in Which Patches and Links Can Contribute to Habitat
Availability and Connectivity in the Landscape.''} \emph{Ecography} 33
(June): 523--37. \url{https://doi.org/10.1111/J.1600-0587.2009.05760.X}.

\bibitem[\citeproctext]{ref-szekely.gabor.2004}
Szekely, Gabor J, and Maria L Rizzo. 2004. {``Testing for Equal
Distributions in High Dimension.''}
\url{https://www.researchgate.net/publication/228918499}.

\bibitem[\citeproctext]{ref-welch.1990}
Welch, William J. 1990. {``Construction of Permutation Tests.''}
\emph{Journal of the American Statistical Association} 85: 693--98.
\url{https://doi.org/10.1080/01621459.1990.10474929}.

\end{CSLReferences}




\end{document}
